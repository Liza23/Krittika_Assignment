\documentclass{article}
\usepackage[utf8]{inputenc}
\usepackage{graphicx}
\usepackage{xcolor}

\title{Binary Star System\\\Large Derivations}
\author{Liza Dahiya}
\date{ }

\begin{document}

\maketitle

\textbf{Consider a binary star system:}\\ \par
Let $m_1$, $r_1$, $v_1$ be respectively the mass, radius of revolution around center of mass and orbital velocity of first binary star and $m_2$, $r_2$, $v_2$ be the mass, radius of revolution around center of mass and orbital velocity of second binary star, respectively. \par
\begin{figure}[h]
    \centering
    \includegraphics[scale=0.1]{circle.jpg}
    \caption{Binary star system}
    \label{fig:mesh1}
\end{figure}
Using the center of mass formula for the two stars, we get,
$$m_1r_1 = m_2r_2$$ 
Since it is a binary star system, the time period of revolution, $T$ for both the stars will be equal. Now using \textbf{Kepler's Third Law}, which is stated as follows:
$$T^2 = \frac{4 \pi^2 r^3}{G m}$$
Here, for the binary stars we have, $r = r_1 + r_2$, and $m = m_1 + m_2$. Hence, for binary stars the above formula translates as: 
$$T^2 = \frac{4 \pi^2 (r_1 + r_2)^3}{G (m_1 + m_2)}$$
Put formula for $m_2$ above equation to calculate mass of star 1,
$$T^2 = \frac{4 \pi^2 (r_1 + r_2)^3}{G (m_1 + \frac{m_1r_1}{r_2})}$$
Take $m_1$ as common and simplify for radius terms,
$$T^2 = \frac{4 \pi^2 r_2 (r_1 + r_2)^2}{G m_1}$$
Now, take $m_1$ on LHS:
$$m_1 = \frac{4 \pi^2 r_2 (r_1 + r_2)^2}{G T^2}$$
Here's the formula for $m_1$ is expressed in terms of $r_1$ and $r_2$, lets rather express it in terms of orbital velocities, using:
$$r_1=\frac{T v_1}{2 \pi} \hspace{1 cm} r_2=\frac{T v_2}{2 \pi}$$
Hence putting these in original equation,
$$m_1 = \frac{4 \pi^2 \frac{T v_2}{2 \pi} (\frac{T v_1}{2 \pi} + \frac{T v_2}{2 \pi})^2}{G T^2}$$
Factoring out common terms, we get $m_1$ as:
{\color{blue} $$m_1 = \frac{T v_2 (v_1 + v_2)^2}{2 \pi G}$$}
And similarly, we get $m_2$ as:
{\color{blue}$$m_2 = \frac{T v_1 (v_1 + v_2)^2}{2 \pi G}$$}
Lets also calculate the velocity of center of mass using this formulas,
$$v_{cm} = \frac{m_1v_1 + m_2v_2}{m_1 + m_2}$$
Lets put the masses calculated in above formula,
$$v_{cm} = \frac{\frac{T v_2 (v_1 + v_2)^2}{2 \pi G}v_1 + \frac{T v_1 (v_1 + v_2)^2}{2 \pi G}v_2}{\frac{T v_2 (v_1 + v_2)^2}{2 \pi G} + \frac{T v_1 (v_1 + v_2)^2}{2 \pi G}}$$
Cancelling out common terms, we get:
{\color{blue}$$v_{cm} = \frac{2 v_1v_2}{v_1 + v_2}$$}

\end{document}
